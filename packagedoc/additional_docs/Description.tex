% --------------------------------------------------------------------------------------------------------------
%
% Copyright 2020-2022 Robert Bosch GmbH

% Licensed under the Apache License, Version 2.0 (the "License");
% you may not use this file except in compliance with the License.
% You may obtain a copy of the License at

% http://www.apache.org/licenses/LICENSE-2.0

% Unless required by applicable law or agreed to in writing, software
% distributed under the License is distributed on an "AS IS" BASIS,
% WITHOUT WARRANTIES OR CONDITIONS OF ANY KIND, either express or implied.
% See the License for the specific language governing permissions and
% limitations under the License.
%
% --------------------------------------------------------------------------------------------------------------

\section{Meaning of "Test Suites Management"}

In the scope of the \rfwcore\ a test suite is either a single robot file containing one or more test cases, or a set of several robot files.

Usually all test cases of a test suite run under the same conditions - but these conditions may be different. For example the same test case is used
to test several different variants of a system under test. Every variant requires individual values for certain configuration parameters.

Tests are carried out at several test benches. All test benches have different hardware configurations.
Also the different test benches may require individual values for configuration parameters used in the tests.

\textbf{Therefore the same tests have to run under different conditions!}

The \rfwcore\ provides several places to define parameters: robot files, resource files, parameter files. But these parameters
are fixed. Therefore we need a more dynamic way of accessing parameters. And we postulate the following: When switching between
tests of several variants and test executions on several test benches, no changes shall be required within the test code.

The outcome is that another position has to be introduced to store values for variant and test bench specific parameters.
And a possibility has to be provided to dynamically make either the one or the other set of values vailable during the execution of
tests - depending on outer circumstances like "\textit{which variant?}" and "\textit{which test bench?}".
Those dynamic configuration values are stored within separate configuration files in JSON format and the \pkg\ makes the values 
available globally during the test execution.

Two different kinds of JSON configuration files are involved:

\begin{enumerate}
   \item \emph{parameter configuration files}

   These configuration files contain all parameter definitions (can be more than one configuration file in a project)

   \item \emph{variant configuration file}

   This is a single configuration file containing the mapping between the several parameter configuration files and a name
   (usually the name of a variant). This name can be used in command line to select a certain parameter configuration file
   containing the values for this variant.

   Background: It's easier simply to use a name for referencing a certain variant instead of having the need always to mention
   the path and name of a configuration file.
\end{enumerate}

To realize a concrete test suites management for your project, you need to

\begin{itemize}
   \item identify the parameters that are variant specific, depending on the number of variants in your project,
   \item identify the parameters that are test bench specific, depending on the number of test benches in your project,
   \item identify the parameters that are both: variant specific and test bench specific,
   \item identify the parameters that have the same value in all variants and test benches.
\end{itemize}

After this

\begin{itemize}
   \item for every set of parameters (variant specific and bench specific) you have to introduce a certain parameter configuration file,
   \item in the variant configuration file you have to define for every variant a variant name together
         with the path to the corresponding parameter configuration file.
\end{itemize}

The content of these configuration files is described in the next section.

% --------------------------------------------------------------------------------------------------------------

\section{Content of configuration files}

\textbf{1. variant configuration file}

This file configures the access to all variant dependent \plog{robot_config*.json} files.

\begin{pythoncode}
{
  "default": {
               "name": "robot_execution_config.json",
               "path": ".../config/"
             },
  "variant_1": {
                 "name": "robot_config_variant_1.json",
                 "path": ".../config/"
               },
  "variant_2": {
                 "name": "robot_config_variant_2.json",
                 "path": ".../config/"
               },
  "variant_3": {
                 "name": "robot_config_variant_3.json",
                 "path": ".../config/"
               }
}
\end{pythoncode}

The example above contains definitions for three variants with names:\\
\pcode{variant_1}, \pcode{variant_2} and \pcode{variant_3}. Additionally a variant named \pcode{default} is defined.
This default configuration becomes active in case of no certain variant name is provided when the test suite is being executed.

Another aspect is important: the \textbf{three dots}.
The path to the \plog{robot_config*.json} files depends on the test file location. A
different number of \plog{../} is required dependent on the directory depth of the test
case location.

Therefore we use here three dots to tell the \pkg\ to search from the test
file location up till the \plog{robot_config*.json} files are found:

\begin{pythonlog}
./config/robot_config.json
../config/robot_config.json
../../config/robot_config.json
../../../config/robot_config.json
\end{pythonlog}

and so on.

Hint: The paths to the \plog{robot_config*.json} files are relative to the position of the test suite - \textbf{and not relative to the position of the
mapping file in which they are defined!} You are free to move your test suites one or more level up or down in the file system, but using the
three dots notation enables you to let the position of the \plog{config} folder unchanged.

It is of course still possible to use the standard notation for relative paths:

\begin{pythoncode}
"path": "./config/"
\end{pythoncode}

\newpage

\textbf{2. parameter configuration files}

In these configuration files all parameters are defined, that shall be available globally during test execution.

Some parameters are required. Optionally the user can add own ones. The following example shows the smallest version
of a parameter configuration file containing only the most important parameters. This version is a default version and part of the
\pkg\ installation.

\begin{pythoncode}
{
  "WelcomeString"   : "Hello... Robot Framework is running now!",
  "Maximum_version" : "1.0.0",
  "Minimum_version" : "0.6.0",
  "Project"         : "RobotFramework Testsuites",
  "TargetName"      : "Device_01"
}
\end{pythoncode}

\pcode{Project}, \pcode{WelcomeString} and \pcode{TargetName} are simple strings that can be used anyhow. \pcode{Maximum_version}
and \pcode{Minimum_version} are part of a version control mechanism: In case of the version of the currently installed
\rfwcore\ is outside the range between \pcode{Minimum_version} and \pcode{Maximum_version}, the test execution stops
with an error message.

The version control mechanism is optional. In case you do not need to have your tests under version control, you can set 
the versions to the value \pcode{null}.

\begin{pythoncode}
"Maximum_version" : null,
"Minimum_version" : null,
\end{pythoncode}

As an alternative it is also possible to remove \pcode{Minimum_version} and \pcode{Maximum_version} completely.

In case you define only one single version number, only this version number is considered. The following combination
makes sure, that the installed \rfwcore\ at least is of version 0.6.0, but there is no upper version limit:

\begin{pythoncode}
"Maximum_version" : null,
"Minimum_version" : "0.6.0",
\end{pythoncode}

Hint: The parameters are keys of an internal configuration dictionary. They have to be accessed in the following way:

\begin{robotcode}
Log    Maximum_version : ${CONFIG}[Maximum_version]
Log    Project : ${CONFIG}[Project]
\end{robotcode}

\vspace{1ex}

The following example is an extended version of a configuration file containing also some user defined parameters.

\begin{pythoncode}
{
  "WelcomeString"   : "Hello... Robot Framework is running now!",
  "Maximum_version" : "1.0.0",
  "Minimum_version" : "0.6.0",
  "Project"         : "RobotFramework Testsuites",
  "TargetName"      : "Device_01"
  "params": {
              // global parameters
              "global" : {
                           "param1" : "ABC",
                           "param2" : 25
                         }
            }
}
\end{pythoncode}

User defined parameters have to be placed inside \pcode{params:global}. The intermediate level \pcode{global} is introduced to enable further
parameter scopes than \pcode{global} in future.

All user defined parameters have the scope \pcode{params:global} per default. Therefore they can be accessed directly:

\begin{robotcode}
Log    param1 : ${param1}
\end{robotcode}

And another feature can be seen in the example above:\\
In the context of the \pkg\ the JSON format is an extended one.
Deviating from JSON standard it is possible to comment out lines with starting them with a double slash \pcode{//}.
This allows to add explanations about the meaning of the defined parameters already within the JSON file.

% --------------------------------------------------------------------------------------------------------------
\newpage

\section{Access to configuration files}

With an installed \pkg\ every test execution requires a configuration - that is the accessibility
of a configuration file in JSON format. The \pkg\ provides four different possibilities - also
called \textit{level} - to realize such an access. These possibilities are sorted and the \pkg\ tries to access
the configuration file in a certain order: Level 1 has the highest priority and level 4 has the lowest priority.

\textbf{Level 1}

Path and name of a parameter configuration file is provided in command line of the \rfwcore.

\textbf{Level 2 (recommended)}

The name of the variant is provided in command line of the \rfwcore.

This level requires that a variant configuration file is passed to the suite setup of the \pkg.

\textbf{Level 3}

The \pkg\ searches for parameter configuration files within a folder \plog{config} in current test suite folder.
In case of such a folder exists and parameter configuration files are inside, they will be used.

\textbf{Level 4 (unwanted, fallback solution only)}

The \pkg\ uses the default configuration file that is part of the installation.

\vspace{2ex}

\textbf{Summary}

\begin{itemize}
   \item With highest priority a parameter configuration file provided in command line, is considered - even in case of also other
configuration files (level 2 - level 4) are available.
   \item If a parameter configuration file is not provided in command line, but a variant name, then the configuration belonging
to this variant, is loaded - even in case of also other configuration files (level 3 - level 4) are available.
   \item If nothing is specified in command line, then the \pkg\ tries to find parameter configuration files within a \plog{config} folder
and take them if available - even in case of also the level 4 configuration file is available.
   \item In case of the user does not provide any information about parameter configuration files to use, the \pkg\ loads the
default configuration from installation folder (fallback solution; level 4).
\end{itemize}

\textbf{In this context two aspects are important to know for users:}

\begin{enumerate}
   \item \textit{Which parameter configuration file is selected for the test execution?}\\
To answer this question the log file contains the path and the name of the selected parameter configuration file.
   \item \textit{For which reason is this parameter configuration file selected?}\\
To answer this question the log file also contains the level number. The level number indicates the reason.
\end{enumerate}

With these log file entries the test execution is clearly understandable, traceable and scales for huge test suites.

\vspace{2ex}

\textbf{Why is level 2 the recommended one?}

Level 2 is the most flexible and extensible solution. Because the robot files contain a link to a variants configuration file,
the possible sets of parameter values can already be taken out of the code.

The values selected by level 1, you only see in the log files, but not in the code, because the selection happens in command line only.

Level 3 has a rather strong binding between robot files and configuration files. If you start the test implementation based on level 3
and after this want to have a variant handling, then you have to switch from level 3 to level 2 - and this causes effort in implementation.

Wherease if you start with level 2 immediately and need to consider another set of configuration values for the same tests, then you only have to add
another parameter configuration file and another entry in the variants configuration file, without changing any test implementation.

\textit{We strongly recommend not to mix up several different configuration levels in one project!}

% --------------------------------------------------------------------------------------------------------------

\section{Activation of "Test Suites Management"}

To activate the test suites management you have to import the \pkg\ library in the following way:

\begin{robotcode}
Library    RobotFramework_TestsuitesManagement    WITH NAME    tm
\end{robotcode}

We recommend to use the \rcode{WITH NAME} option to shorten the robot code a little bit.

The next step is to call the \rcode{testsuite_setup} of the \pkg\ within the \rcode{Suite Setup} of your test:

\begin{robotcode}
Suite Setup    tm.testsuite_setup
\end{robotcode}

As long as you
\begin{itemize}
   \item do not provide a parameter configuration file in command line when executing the test suite (level 1),
   \item do not provide a variants configuration file as parameter of the \rcode{testsuite_setup} (level 2),
   \item do not have a \plog{config} folder containing parameter configuration files in your test suites folder (level 3),
\end{itemize}
the \pkg\ falls back to the default configuration (level 4).

In case you want to realize a variant handling you have to provide the path and the name of a variants configuration file to the \rcode{testsuite_setup}:

\begin{robotcode}
Suite Setup    tm.testsuite_setup    ./config/exercise_variants.json
\end{robotcode}

To ease the analysis of a test execution, the log file contains informations about the selected level and the path and the name of the used
configuration file, for example:

\begin{robotlog}
Running with configuration level: 2
CfgFile Path: ./config/exercise_config.json
\end{robotlog}

Please consider: The \rcode{testsuite_setup} requires a variants configuration file (in the example above: \plog{exercise_variants.json}) - whereas
the log file contains the resulting parameter configuration file (in the example above: \plog{exercise_config.json}), that is selected depending
on the name of the variant provided in command line of the \rfwcore.

% --------------------------------------------------------------------------------------------------------------
\newpage

\section{Variants selection}

In a previous section the level concept for configuration files has been explained. This section contains corresponding code examples.

\textit{1. Selection of a certain parameter configuration file in command line}

\begin{robotlog}
--variable config_file:"(path to parameter configuration file)"
\end{robotlog}

\textit{2. Selection of a certain variant per name in command line}

\begin{robotlog}
--variable variant:"(variant name)"
\end{robotlog}

\textit{3. Parameter configuration taken from} \rlog{config} \textit{folder}

This \rlog{config} folder has to be placed in the same folder than the test suites.

Parameter configuration files within this folder are considered under two different conditions:

\begin{itemize}
   \item The configuration file has the name \rlog{robot_config.json}. That is a fix name predefined by the \pkg.
   \item The configuration file has the same name than a robot file inside the test suites folder, e.g.:
         \begin{itemize}
            \item Name of test suite file: \rlog{example.robot}
            \item Path and name of corresponding parameter configuration file: \rlog{./config/example.json}
         \end{itemize}
         With this rule it is possible to give every test suite in a certain folder an own individual configuration.
\end{itemize}

% --------------------------------------------------------------------------------------------------------------
\newpage

\section{Local configuration}

It might be required to execute tests on several different test benches with every test bench has it's own individual hardware
that might require configuration parameter values that are test bench specific. This can be related to common configuration parameters
and also to parameters that are variant specific. In the second case a configuration parameter is both variant specific \textit{and}
test bench specific.

The \textit{local configuration} feature of the \pkg\ provides the possibility to define test bench specific configuration parameter values.

The meaning of \textit{local} in this context is: placed on a certain test bench - and valid for this bench only.

Also this local configuration is based on configuration files in JSON format. These files are the last ones that are considered when the configuration is loaded.
The outcome is that it is possible to define default values for test bench specific parameters in other configuration files - to be also test bench independent.
And it is possible to use the local configuration to overwrite these default values with values that are specific for a certain test bench.

\textbf{Important:}

\begin{itemize}
   \item Local configuration files are fragments only - and not a full configuration! Even so they need to follow the JSON syntax rules.
         This means, at least they have to start with an opening curly bracket and they have to end with a closing curly bracket.
   \item Local configuration files must not contain the mandatory top level parameters like the \pcode{WelcomeString} and others.
\end{itemize}

Using the local configuration feature is an option and the \pkg\ provides two ways to realize it:

\begin{enumerate}
   \item \textit{per command line}

         Path and name of the local parameter configuration file is provided in command line of the \rfwcore\ with the following syntax:
\begin{robotlog}
--variable local_config:"(path to local configuration file)"
\end{robotlog}

   \item \textit{per environment variable}

         An environment variable named \rcode{ROBOT_LOCAL_CONFIG} exists and contains path and name of a local parameter configuration file.

         \textbf{The user has to create this environment variable!}

         This mechanism allows a user - without any command line extensions - automatically to refer on every test bench to an individual local configuration,
         simply by giving on every test bench this environment variable an individual value.

\end{enumerate}

The command line has a higher priority than the environment variable. If both is available the local configuration is taken from command line.

Recommendation: \textit{To avoid an accidental overwriting of local configuration files in version control systems we recommend to give those files
names that are test bench specific.}

% --------------------------------------------------------------------------------------------------------------
\newpage

\section{Priority of configuration parameters}

In previous sections the level concept has been explained. This concept introduces four levels of priority
that define, which of the possible sources of configuration parameters are processed.
But there are other rules involved that influence the priority:

\begin{itemize}
   \item The local configuration has higher priority than other parameter configurations
   \item The command line has higher priority than definitions within configuration files
\end{itemize}

Already in command line we have several possibilities to make settings:

\begin{itemize}
   \item Set a parameter configuration file (with \pkg\ command line variable \rlog{config_file}, \textit{level 1})
   \item Set a variant name (with \pkg\ command line variable \rlog{variant}, \textit{level 2})
   \item Set a local configuration (with \pkg\ command line variable \rlog{local_config})
   \item Set any other variables (directly with \rfwcore\ command line variable \rlog{--variable})
\end{itemize}

And it is possible that in all four use cases the same parameters are used. Or in other words: It is possible to use the
\rlog{--variable} mechanism to define a parameter that is also defined within a parameter configuration or within a
local configuration - or in both together.

\vspace{2ex}

Finally this is the order of processing (with highest priority first):

\begin{enumerate}
   \item Single command line variable (\rlog{--variable})
   \item Local configuration (\rlog{local_config})
   \item Variant specific configuration (\rlog{config_file} or \rlog{variant})
\end{enumerate}

Meaning:

\begin{enumerate}
   \item Variant specific configuration is overwritten by local configuration
   \item Local configuration is overwritten by single command line variable
\end{enumerate}

What happens in case of a command line contains both a \rlog{config_file} and a \rlog{variant}?

\rlog{config_file} is level 1 and \rlog{variant} is level 2. Level 1 has higher priority than level 2. Therefore \rlog{config_file}
is the valid one. This does \textbf{not} mean that \rlog{config_file} overwrites \rlog{variant}! In case of a certain level is identified
(here: level 1), all other levels are ignored. The outcome is that - in this example - the \rlog{variant} has no meaning.
Between different levels there is an \textit{either or} relationship. And that is the reason for that it makes no sense to define both in command line,
a \rlog{config_file} and a \rlog{variant}. The \pkg\ throws an error in this case.

But when additionally \rlog{--variable} is used to define a new value for a parameter that is already defined in one of the involved configuration files,
then the configuration file value is overwritten by the command line value.

% --------------------------------------------------------------------------------------------------------------
\newpage

\section{Nested configuration files}

In case of a project requires more and more parameters, it makes sense to split the growing configuration file into smaller ones.

This means, at first we have to split all configuration parameters in
\begin{enumerate}
   \item parameters that are specific for a certain variant,
   \item common parameters that have the same value for all variants
\end{enumerate}

Placing those common parameters in every single variant specific parameter configuration file would craeate a lot of redundancy.
This would also complicate the maintenance.

The solution is to use the variant specific configuration files only for variant specific parameters and to put all common parameters in
a separate configuration file. This common parameter file has to be imported in every variant specific parameter file.

The outcome is that still with the selection of a certain variant specific parameter file both types of parameters are available:
the variant specific ones and the common ones.

This can be done in the following way:

For example we have the following variant specific configuration files:

\begin{pythonlog}
config/config_variant1.json
config/config_variant2.json
\end{pythonlog}

Additionaly we have a configuration file with common parameters:

\begin{pythonlog}
config/config_common.json
\end{pythonlog}

The import of \rlog{config_common.json} into \rlog{config_variant1.json} and into \rlog{config_variant2.json} is possible in the following way:

\begin{pythonlog}
"params" : {
            "global": {
                       "[import]"   : "./config_common.json",
                       "teststring" : "variant specific value"
                      }
           }
\end{pythonlog}

The key \pcode{[import]} indicates the import of another configuration file. The value of the key is the path and name of this file.

Imports can be nested. An imported configuration file is allowed to contain imports also.

The content of the importing file and the content of all imported files are merged. In case of duplicate parameter names follow up definitions
overwrite previous definitions of the same parameter!

\textbf{Important:}

\begin{itemize}
   \item All imported configuration files are fragments only - and not a full configuration! Even so they need to follow the JSON syntax rules.
         This means, at least they have to start with an opening curly bracket and they have to end with a closing curly bracket.
   \item Imported configuration files must not contain the mandatory top level parameters like the \pcode{WelcomeString} and others.
\end{itemize}

% --------------------------------------------------------------------------------------------------------------
\newpage

\section{Overwritten parameters}

Summarized the \pkg\ provides three different types of parameter configuration files to define parameters:

\begin{enumerate}
   \item A full standard parameter configuration file containing at least the mandatory parameters and - as option - also user defined parameters
   \item A parameter configuration file fragment that is imported in other configuration files by the \pcode{[import]} key
   \item A local parameter configuration file that is also a fragment only, and accessed either by command line or environment variable
\end{enumerate}

All types of configuration file can be used

\begin{enumerate}
   \item to define new parameters
   \item to overwrite already existing parameters
\end{enumerate}

This possibility only belongs to user defined parameters with scope \pcode{params:global}!

\vspace{2ex}

Example:

\textit{1. Define a new parameter:}

\begin{pythoncode}
"params" : {
            "global": {
                       "teststring" : "initial value"
                      }
           }
\end{pythoncode}

\vspace{2ex}

\textit{2. Overwrite an already existing parameter:}

To overwrite a parameter is - after the initial definition - possible at any follow up position

\begin{itemize}
   \item in the same configuration file or
   \item in other configuration files like the imported ones or
   \item in a local configuration file
\end{itemize}

With the following syntax:

\begin{pythoncode}
${params}['global']['teststring'] : "new value"
\end{pythoncode}

The resulting value of a parameter at the end depends on the priority (computation order) described in previous sections of this description.

% --------------------------------------------------------------------------------------------------------------
\newpage

\section{Tutorials}

What is described up to here can be experienced in the tutorial \plog{900_building_testsuites}.

It is also recommended to take a look at the tutorial \plog{100_variables_and_datatypes}.
This tutorial goes more into detail about data types and explaines how to handle also other data types like strings in
configuration files of the \pkg.



