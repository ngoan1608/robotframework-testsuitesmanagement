% --------------------------------------------------------------------------------------------------------------
%
% Copyright 2020-2022 Robert Bosch GmbH

% Licensed under the Apache License, Version 2.0 (the "License");
% you may not use this file except in compliance with the License.
% You may obtain a copy of the License at

% http://www.apache.org/licenses/LICENSE-2.0

% Unless required by applicable law or agreed to in writing, software
% distributed under the License is distributed on an "AS IS" BASIS,
% WITHOUT WARRANTIES OR CONDITIONS OF ANY KIND, either express or implied.
% See the License for the specific language governing permissions and
% limitations under the License.
%
% --------------------------------------------------------------------------------------------------------------

The \pkg\ enables users to define dynamic configuration values within separate configuration files in JSON format.

These configuration values are available during test execution - but under certain conditions that can be defined by the user
(e.g. to realize a variant handling). This means: Not all parameter values are available during test execution - only the ones
that belong to the current test scenario.

To realize this, the \pkg\ provides the following features:

\begin{enumerate}
   \item Split all possible configuration values into several JSON configuration files, with every configuration file contains a specific
         set of values for configuration parameter
   \item Use nested imports of JSON configuration files
   \item Follow up definitions in configuration files overwrite previous definitions (of the same parameter)
   \item Select between several criteria to let the Robot Framework use a certain JSON configuration file
\end{enumerate}

\vspace{2ex}

\textbf{How to install}

\vspace{2ex}

The \pkg\ can be installed in two different ways.

\begin{enumerate}

\item Installation via PyPi (recommended for users)
\begin{pythonlog}
pip install RobotFramework_TestsuitesManagement
\end{pythonlog}

\href{https://pypi.org/project/robotframework-testsuitesmanagement}{\pkg\ in PyPi}

\item Installation via GitHub (recommended for developers)

\begin{itemize}
\item Clone the \repo\ repository to your machine.
\begin{pythonlog}
git clone https://github.com/test-fullautomation/robotframework-testsuitesmanagement.git
\end{pythonlog}

\href{https://github.com/test-fullautomation/robotframework-testsuitesmanagement}{\pkg\ in GitHub}

\vspace{2ex}

\item Use the following command to install \pkg:
\begin{pythonlog}
setup.py install
\end{pythonlog}

\end{itemize}
\end{enumerate}
